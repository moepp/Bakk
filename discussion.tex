%%%% Time-stamp: <2012-08-20 17:41:39 vk>

%% example text content
%% scrartcl and scrreprt starts with section, subsection, subsubsection, ...
%% scrbook starts with part (optional), chapter, section, ...
\chapter{Diskussion}
\label{chap:disc}

Das folgende Kapitel geht auf die Vorteile und auch auf Nachteile der entwickelten App ein und diskutiert
Erweiterungsmöglichkeiten der App, die im Rahmen der Entwicklung nicht berücksichtigt wurden. 

Der Fokus der Entwicklung, der in dieser Arbeit vorgestellten App sollte deutlich erkennbar sein, nämlich
die Klarheit und Einfachheit in der Bedienung und Darstellung der Inhalte. Die Umsetzung der App in der 
beschriebenen Art und Weise ist mehreren Tatsachen geschuldet. Zum einen gelten als Adressaten der 
App vor allem Kinder im Volksschulalter, die entweder gerade beim Erlernen der Grundrechenarten sind,
oder Defizite dabei aufweisen, und zum anderen die mobile Plattform selbst. Da SchülerInnen oft schon zu 
rechnen beginnen bevor sie überhaupt lesen können, wurde der Aufbau und das Design der App so gewählt,
dass man eigentlich auch ohne lesen zu können die App bedienen kann. Um Kinder zu ermutigen sich mit 
dieser App zu befassen wurde auch der Fokus im Design auf kindliche Symbole und Grafiken gelegt. 
Hierbei sei erwähnt, dass die grafische Gestaltung der App von zwei Schülerinnen der Ortweinschule Graz entwickelt wurde.
Zum anderen gelten bei der Erstellung von Apps für Apples mobiles Betriebssystem iOS gewisse 
Richtlinien\footcite{https://developer.apple.com/library/ios/documentation/userexperience/conceptual/mobilehig/}
die einzuhalten sind. Diese schränken den Entwickler ein, zu komplexe User Interfaces zu entwerfen, 
um so die Bedienung der App einfach und konsistent über alle Bildschirme zu halten.

Der Additions- und Subtraktionstrainer dient einzig und allein der Darstellung von Rechenaufgaben. Diese
kommen wie bereits in Abschnitt \ref{subsec:trainer} erwähnt entweder vom Webservice, oder
werden, wie Abschnitt \ref{sec:excercise} erläutert, von der App nach gewissen Kriterien generiert.
Als Erweiterung dieser App, wäre es durchaus denkbar einen Menüpunkt einzubauen, über den
man eine Statistik einsehen kann. Denkbar hierbei wäre zum Beispiel anzuzeigen, wieviel Prozent 
aller gestellten Rechenaufgaben vom Schüler korrekt gelöst wurden, oder weiters eine grafische
Darstellung wieviel Prozent der Rechenaufgaben mit steigendem Schwierigkeitsgrad erfüllt wurden. 
Abbildung \ref{fig:success_calculations} zeigt wie so ein Diagramm aussehen könnte.

\myfig{success_calculations}%% filename in figures folder
  {width=1\textwidth,height=1\textheight}%% maximum width/height, aspect ratio will be kept
  {Statistik über die Leistung des/der SchülerIn}%% caption
  {Statistik über die Leistung des/der SchülerIn}%% optional (short) caption for table of figures
  {fig:success_calculations}%% label


Vorstellbar wäre auch ein textueller Hinweis(zB. \enquote{Du solltest zweistellige Subtraktionen noch 
etwas üben denn du weißt ja, Übung macht den Meister}), der auf Wunsch in der App angezeigt wird, über die
Schwerpunkte die sich ein/eine SchülerIN beim Üben legen sollte. Um diese gerade diskutierten Erweiterungen 
umzusetzen Bedarf es jedoch Erweiterungen am Grund-System, denn wie bereits
erwähnt speichert die App keinen Verlauf der Rechenerfolge des/der BenutzerIn. Da diese Statistiken im 
Sytem jedoch vorliegen, dürfte es keine größeren Schwierigkeiten bereiten weitere Schnittstellen
des Webservice für die App bereitzustellen.
Eine weitere vorstellbare Ergänzung in der App wäre eine Rangliste für den Übungsmodus.


%% vim:foldmethod=expr
%% vim:fde=getline(v\:lnum)=~'^%%%%\ .\\+'?'>1'\:'='
%%% Local Variables: 
%%% mode: latex
%%% mode: auto-fill
%%% mode: flyspell
%%% eval: (ispell-change-dictionary "en_US")
%%% TeX-master: "main"
%%% End: 
