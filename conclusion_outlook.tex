%%%% Time-stamp: <2012-08-20 17:41:39 vk>

%% example text content
%% scrartcl and scrreprt starts with section, subsection, subsubsection, ...
%% scrbook starts with part (optional), chapter, section, ...
\chapter{Zusammenfassung und Ausblick}
\label{chap:concl}

In diesem Kapitel wird noch einmal kurz zusammengefasst, was in dieser Arbeit beschrieben wurde.
Bevor man sich der Aufgabe stellt eine mobile Applikation für ein Smartphone zu entwickeln, sollte
man sich zuerst einen Überblick verschaffen, ob es, und in welcher Form es bereits Arbeiten zu 
diesem Thema gibt. Dieser Anforderung wurde in Kapitel \ref{chap:sota} Rechnung getragen, und einige
ähnlich Arbeiten recherchiert.
In Kapitel \ref{chap:impl} wurden die technischen Details der Entwicklung erläutert. 
Der Schwerpunkt dieser Arbeit lag in der Entwicklung der App, warum dieses Kapitel auch das umfangreichste
in dieser Arbeit darstellt, wobei zum Verständnis der Applikation Abbildung \ref{fig:communicationchart}
auf Seite \pageref{fig:communicationchart} einen guten Überblick darstellt. In diesem Kapitel findet
man auch Screenshots zur entwickelten App.
Mögliche Ergänzungen zur App wurden in Kapitel \ref{chap:disc} diskutiert.


Die Anwendung von Technologien im mobilen Bereich steckt noch immer in ihren Kinderschuhen, 
nimmt aber auch gleichzeitig immer mehr Fahrt auf, wie es auch die Arbeit von ~\cite{Ebner2013} zeigt. 
Aufgrund des großen Potenzials von \enquote{Mobile Learning}, können wir uns erwarten, 
dass die Entwicklungsgeschwindigkeit in diesem Forschungsgebiet weiter rapide steigen wird.

Um nun einen Ausblick auf die Entwicklung des \enquote{Mobile Learning} zu wagen, sollte 
man sich zuerst kurz überlegen, was man unter \enquote{Mobile Learning} eigentlich
verstehen sollte. In der Arbeit von~\cite{Brown2010} wurde eine eine Einteilung von \enquote{Mobile Learning}
Geräten gemacht, die wie folgt aussieht: 
\begin{description}
	\item[Highly Mobile Device] Gerät in der Größe eines Mobiltelefons, das man in der Hosentasche
	unterbringen kann: \enquote{Feature Phones} (reine Sprach und Text Services), Smartphones und andere ähnliche Geräte
	\item[Very Mobile Device] Netbooks, Pads, Slates
	\item[Mobile Device] Größere Geräte wie Notebooks
\end{description}

Die Verwendung des Begriffs \enquote{Mobile Learning} wird aufgrund der Tatsache, dass es sich
hierbei um eine Smartphone App handelt, auf \enquote{Highly Mobile Devices} bezogen.
Es ist vorstellbar, dass \enquote{Mobile Learning} gerade in Entwicklungsländern mit ihren sehr
jungen Alterstrukturen aufgrund vieler Kinder, seinen großen Durchbruch erleben könnte. Hierbei zu
erwähnen sind diverse Initiativen von großen Hardwareherstellern wie Microsoft \footcite{http://www.epo.de/index.php?option=com_content&view=article&id=1809:microsoft-will-smartphone-fuer-entwicklungslaender-herstellen&catid=20:it-a-entwicklung&Itemid=51}
Nokia \footcite{http://www.n24.de/n24/Nachrichten/Netzwelt/d/1471900/nokia-setzt-auf-entwicklungslaender.html} und Huawei \footcite{http://ht4u.net/news/27090_huawei_bringt_windows_phone_4afrika_-_smartphone_fuer_entwicklungslaender/},  
zur Verbreitung von günstigen Smartphones in diesen Ländern.

Die Bildungsqualität leidet dort unter den hohen Schülerzahlen wogegen man mit gezielter Förderung
und auch Forderung durch maßgeschneiderte Rechenbeispiele, wie es die in dieser Arbeit beschriebene
App macht, einen großen Schritt bei der Entwicklung nach vorne machen kann.






%% vim:foldmethod=expr
%% vim:fde=getline(v\:lnum)=~'^%%%%\ .\\+'?'>1'\:'='
%%% Local Variables: 
%%% mode: latex
%%% mode: auto-fill
%%% mode: flyspell
%%% eval: (ispell-change-dictionary "en_US")
%%% TeX-master: "main"
%%% End: 
