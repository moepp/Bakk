%%%% Time-stamp: <2013-09-24 14:39:01 vk>
%% ========================================================================
%%%% Disclaimer
%% ========================================================================
%%
%% created by
%%
%%      Stephan Moser
%%

%% ========================================================================
%%%% Introduction
%% ========================================================================

%% example text content
%% scrartcl and scrreprt starts with section, subsection, subsubsection, ...
%% scrbook starts with part (optional), chapter, section, ...
\chapter{Einleitung}

Zweifelsfrei hat das Aufkommen Mobiler Technologien und Smartphones mittlerweile große 
Auswirkungen auf unser tägliches Leben. Dazu gehört auch die Art wie wir heutzutage
Lernen. 
Um dieser Tatsache Rechnung zu tragen wurde im Zuge dieser Arbeit eine iPhone Application (kurz App genannt)  für Apple's 
Smartphone Betriebssystem iOS entwickelt, die Benedikt \citeauthor{Neuhold2013} bei seiner Diplomarbeit \enquote{Adaptives Informationssystem zur Erlernung
mehrstelliger Addition und Subtraktion} \footcite{Neuhold2013} unterstützen soll.

Konkret geht es darum, dass es durch diese App für SchülerInnen unkompliziert und schnell möglich sein soll
Additionen und Subtraktionen zu üben. Dazu melden sich die SchülerInnen über die iPhone App bei Neuhold's System an, und 
bekommen daraufhin auf ihre Bedürfnisse angepasste Rechenübungen die ihrem derzeitigen Wissensstand entsprechen.
Der eigentliche Zweck der App besteht aber darin, dass die Ergebnisse und auch 
alle Zwischenergebnisse in Form von Überträgen protokolliert
werden und in weiterer Folge an das bereits erwähnte System von Benedikt
Neuhold zur Analyse weitergeleitet werden.

Da in erster Linie Kinder im Volksschulalter die Adressaten für Addition -und Subtraktionsübungen sind
liegt ein wesentlicher Teil der Arbeit darin, die App so einfach wie möglich und dabei grafisch 
ansprechend zu gestalten, um die langfristige Motivation der SchülerInnen sicherzustellen.

Im folgenden Abschnitt~\ref{sec:Gliederung} wird ein kurzer Überblick auf diese schriftliche Arbeit gegeben.
\section{Gliederung der Arbeit}
\label{sec:Gliederung}

In Kapitel \ref{chap:sota} wird kurz darauf eingegangen, welche Arbeiten es zu diesem Thema bereits 
gibt, und in welcher Form sich diese von der hier diskutierten Arbeit unterscheiden. 

Kapitel \ref{chap:impl} handelt von der technischen Umsetzung der App, das heißt es wird 
beschrieben welche Technologien zur Umsetzung der Arbeit verwendet wurden und wie diese im 
Kontext dieser App angepasst und verwendet wurden.

Gewonnene Ergebnisse sowie aufgetretene Probleme im Vorfeld der Arbeit, während der Umsetzung aber auch in der Nachbereitung
werden in Kapitel \ref{chap:disc} diskutiert.

Das vorletzte Kapitel \ref{chap:concl} fasst die gesamte Arbeit mit all den gewonnenen Erkenntnissen
noch einmal zusammen bevor in Kaptiel \ref{chap:outlook} ein Ausblick gewagt wird in welche 
Richtung sich das Thema des Mobilen Lernens hinentwickeln wird.



%% vim:foldmethod=expr
%% vim:fde=getline(v\:lnum)=~'^%%%%\ .\\+'?'>1'\:'='
%%% Local Variables: 
%%% mode: latex
%%% mode: auto-fill
%%% mode: flyspell
%%% eval: (ispell-change-dictionary "en_US")
%%% TeX-master: "main"
%%% End: 
