%%%% Time-stamp: <2012-08-20 17:41:39 vk>

%% example text content
%% scrartcl and scrreprt starts with section, subsection, subsubsection, ...
%% scrbook starts with part (optional), chapter, section, ...
\chapter{Umsetzung}
\label{chap:impl}

blabla



This is my text with an example Figure~\ref{fig:example} and example
citation~\cite{StrunkWhite} or \textcite{Bringhurst1993}. And there is another
\enquote{citation} which is located at the bottom\footcite{tagstore}.

\myfig{TU_Graz_Logo}%% filename in figures folder
  {width=0.1\textwidth,height=0.1\textheight}%% maximum width/height, aspect ratio will be kept
  {Example figure.}%% caption
  {}%% optional (short) caption for table of figures
  {fig:example}%% label

Now you are able to write your own document. Always keep in mind: it's
the \emph{content} that matters, not the form. But good typography is
able to deliver the content much better than information set with bad
typography. This template allows you to concentrate on writing good
content while the form is done by the template definitions.


%% vim:foldmethod=expr
%% vim:fde=getline(v\:lnum)=~'^%%%%\ .\\+'?'>1'\:'='
%%% Local Variables: 
%%% mode: latex
%%% mode: auto-fill
%%% mode: flyspell
%%% eval: (ispell-change-dictionary "en_US")
%%% TeX-master: "main"
%%% End: 
