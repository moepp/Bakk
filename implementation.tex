%%%% Time-stamp: <2012-08-20 17:41:39 vk>

%% example text content
%% scrartcl and scrreprt starts with section, subsection, subsubsection, ...
%% scrbook starts with part (optional), chapter, section, ...
\chapter{Umsetzung}
\label{chap:impl}

Das folgende Kapitel beschäftigt sich mit der technischen Umsetzung und einer Erläuterung der einzelnen
Module des Additions- und Subtraktionstrainers. Zur Erstellung der App wurde die von Apple integrierte Entwicklungsumgebung 
\enquote{Xcode\footcite{http://developer.apple.com/xcode/ [Zugriff am 24.10.2013]}} in der Version \enquote{4.6.2} verwendet um die App zu schreiben.
Die zu diesem Zweck verwendete Programmiersprache ist zum Großteil 
\enquote{Ojective-C\footcite{http://www.gnu.org/software/gnustep/resources/documentation/Developer/Base/ProgrammingManual/manual_toc.html [Zugriff am 24.10.2013]}} die als
objektorientierte Erweiterung der Programmiersprache \enquote{C} angesehen wird.

Der Aufbau der Addition- und Subtraktionstrainer App wurde sehr modular gestaltet. Durch die Verwendung 
von \enquote{ViewControllern \footcite{http://developer.apple.com/library/ios/documentation/uikit/reference/UIViewController_Class/ [Zugriff am 24.10.2013]}}
lässt sich dieser modulare Aufbau sauber beibehalten, was die Erweiterung der App vereinfacht.

Wegen dieser Modularisierung der App werden in den folgenden Abschnitten die technischen Einzelheiten pro Modul erläutert.
%\section{Module}
%\label{sec:modules}
%In diesem Abschnitt wird jedes Modul der App erläutert. Zu diesem Zweck werden auch Codeausschnitte 
%und Screenshots angeführt und erklärt.
\section{Hauptmenü}
Das Hauptmenü dient als Einstiegspunkt in die App. Das heißt sobald der/die BenutzerIn die App startet
landet er/sie sofort im Hauptmenü. Dieses gilt rein als Ausgangspunkt um die einzelnen Funktionalitäten
der App aufrufen zu können. Das Hauptmenü besteht aus folgendenden Menüpunkten:

\begin{description}
	\item[Trainer] Webbasiertes Trainingssystem für das ein Konto benötigt wird. Abschnitt \ref{subsec:trainer} erläutert dieses Modul. 
	\item[Üben] In unterschiedlichen Schwierigkeitsstufen anonym durchführbare Übung. Abschnitt \ref{subsec:excercise} erläutert dieses Modul.
	\item[Hilfe] Kurzer Hilfetext zur App
	\item[Einstellungen] Einstellungen wie zum Beispiel die Ausrichtung des Übertrages	
\end{description}

Da \enquote{Xcode} eine sehr komfortable Möglichkeit bietet diese Menüpunkte
zu verklinken, hat das Hauptmenü auch keine weitere Funktionalität in Form von Programmcode. 
In Abbildung \ref{fig:mainmenu} sehen sie einen Screenshot des Hauptmenüs. 
\myfig{mainmenu}%% filename in figures folder
  {width=0.4\textwidth,height=0.4\textheight}%% maximum width/height, aspect ratio will be kept
  {Screenshot des Hauptmenüs.}%% caption
  {Screenshot Hauptmenü.}%% optional (short) caption for table of figures
  {fig:mainmenu}%% label

\section{Trainer}
\label{subsec:trainer}
Der Trainer bildet im eigentlichen Sinn das Herzstück der App. Sobald der Trainer gestartet wird, 
wird überprüft ob der/die BenutzerIn bereits Zugriff auf das Neuhold's System hat. Ist dies nicht der Fall
muss sich der/die BenutzerIn über ein Webinterface für die Nutzung des Trainers registrieren. Sobald
die Registrierung erfolgt ist, kann sich der/die Benutzerin am System einloggen. Abbildung \ref{fig:login}
zeigt einen Screenshot der App auf dem die Registrierung bzw. der Login ersichtlich sind.

\myfig{login}%% filename in figures folder
  {width=0.4\textwidth,height=0.4\textheight}%% maximum width/height, aspect ratio will be kept
  {Screenshot der Registrierungs bzw. des Logins.}%% caption
  {Screenshot Login.}%% optional (short) caption for table of figures
  {fig:login}%% label
  
Die gesamte Kommunikation zwischen App und Neuhold's Webservice erfolgt über \enquote{SOAP\footcite{http://www.w3.org/TR/soap9/}} Nachrichten.
Das Nachrichtenformat für diesen Zweck ist \enquote{XML\footcite{http://www.w3.org/XML/}}. \enquote{SOAP} Nachrichten
selbst zu erstellen und an das Webservice zu schicken wäre äußerst umständlich und fehleranfällig, weswegen
hierzu ein Service namens \enquote{SudzC\footcite{http://sudzc.com}} verwendet wurde. 
Dieses Service generiert aufgrund der SOAP Endpoint Definition automatisch Programmcode der beim 
Programmieren der App aufgerufen werden kann und über diesen das Senden von SOAP Nachrichten 
vereinfacht wird.

Folgender Codeausschnitt \ref{listing:authXML} zeigt, wie eine SOAP Nachricht für die Authentifizierung aussieht. Der SOAP
Endpoint für diese Nachricht ist \enquote{isUserAllowed}. Das Feld \enquote{username} wird im Klartext 
übertragen, das Passwort im Feld \enquote{password} ist SHA-256 gehasht(Beschrieben in \cite{NSA2001}). Für die Nachvollziehbarkeit,
auf welchem System der Additions -und Subtraktionstrainer ausgeführt wird, ist im Feld \enquote{idApp} die App ID
der jeweiligen App einzutragen (iOS, Android, etc.). \enquote{hmacClient} ist ein Hash der sich aus Benutzername, gehashtem Password, App ID und App Key zusammensetzt.
\begin{lstlisting}[caption=SOAP XML Nachricht zur Benutzer Authentifizierung, label=listing:authXML]
<soapenv:Envelope xmlns:xsi="http://www.w3.org/2001/XMLSchema-instance" xmlns:xsd="http://www.w3.org/2001/XMLSchema" xmlns:soapenv="http://schemas.xmlsoap.org/soap/envelope/" xmlns:soap="http://mathe.tugraz.at/~georg/Usermanager/public/soap">
   <soapenv:Header/>
   <soapenv:Body>
      <soap:isUserAllowed soapenv:encodingStyle="http://schemas.xmlsoap.org/soap/encoding/">
         <username xsi:type="xsd:string">?</username>
         <password xsi:type="xsd:string">?</password>
         <idApp xsi:type="xsd:int">?</idApp>
         <hmacClient xsi:type="xsd:string">?</hmacClient>
      </soap:isUserAllowed>
   </soapenv:Body>
</soapenv:Envelope>\end{lstlisting} 
Codeausschnitt \ref{listing:hash} zeigt, wie der Hash für die Authentifizierung erstellt wird. Codezeilen
20 und 21 zeigen den Zugriff auf die durch \enquote{SudzC} generierten Methoden zur SOAP Kommunikation.
\begin{lstlisting}[caption=Objective-C Code zur Erstellung des Hashs, label=listing:hash]
        const char *s = [pw cStringUsingEncoding:NSASCIIStringEncoding];
        NSData *pwdata = [NSData dataWithBytes:s length:strlen(s)];
        uint8_t digest[CC_SHA256_DIGEST_LENGTH] = {0};
        CC_SHA256(pwdata.bytes, pwdata.length, digest);
        NSData *out = [NSData dataWithBytes:digest length:CC_SHA256_DIGEST_LENGTH];
        NSString *hash = [out description];
        hash = [hash stringByReplacingOccurrencesOfString:@" " withString:@""];
        hash = [hash stringByReplacingOccurrencesOfString:@"<" withString:@""];
        hash = [hash stringByReplacingOccurrencesOfString:@">" withString:@""];
        NSString *data = [NSString stringWithFormat:@"%@%@%@", uname, hash, @"12"];
        const char *cKey = [key cStringUsingEncoding:NSASCIIStringEncoding];
        const char *cData = [data cStringUsingEncoding:NSASCIIStringEncoding];
        unsigned char cHMAC[CC_SHA256_DIGEST_LENGTH];
        CCHmac(kCCHmacAlgSHA256, cKey, strlen(cKey), cData, strlen(cData), cHMAC);
        NSData *outhmac = [NSData dataWithBytes:cHMAC length:CC_SHA256_DIGEST_LENGTH];
        NSString *hmachash = [outhmac description];
        hmachash = [hmachash stringByReplacingOccurrencesOfString:@" " withString:@""];
        hmachash = [hmachash stringByReplacingOccurrencesOfString:@"<" withString:@""];
        hmachash = [hmachash stringByReplacingOccurrencesOfString:@">" withString:@""];
        adSubUsermanager_Soap_ManagementService *userManagerService = [[adSubUsermanager_Soap_ManagementService alloc] init];
        [userManagerService isUserAllowed:self action:@selector(handleUMService:) username:uname password:hash idApp:12 hmacClient:hmachash];
\end{lstlisting} 

Bei erfolgreicher Authentifizierung wird der/die BenutzerIn nochmals auf einen Zwischenscreen geleitet auf 
dem die Möglichkeit besteht sich wieder abzumelden bzw. mit dem Training zu beginnen. Abbildung \ref{fig:logout} zeigt
den korrespondierenden Screenshot. Ergänzend sei hier erwähnt, dass die App sich den aktuellen Benutzer merkt, und diesen automatisch einloggt bis zu
dem Zeitpunkt an dem sich der Benutzer vom System manuell über den Menüpunkt \enquote{Abmelden} vom System 
ausloggt. Ob der/die BenutzerIn eingeloggt ist wird über \enquote{NSUserDefaults \footcite{https://developer.apple.com/library/mac/documentation/cocoa/Reference/Foundation/Classes/NSUserDefaults_Class/Reference/Reference.html}}
gespeichert.
\myfig{logout}%% filename in figures folder
  {width=0.4\textwidth,height=0.4\textheight}%% maximum width/height, aspect ratio will be kept
  {Screenshot des Logouts}%% caption
  {Screenshot Logout.}%% optional (short) caption for table of figures
  {fig:logout}%% label
  
Beim Start des Trainers wird die Rechenaufgabe wie in Abbildung \ref{fig:trainer} dargestellt. Auf diesem
Screen wird dem/der BenutzerIn ein an seine/ihre Bedürfnisse bzw. Kenntnisse angepasste Rechenaufgabe
dargestellt.
\myfig{trainer}%% filename in figures folder
  {width=0.4\textwidth,height=0.4\textheight}%% maximum width/height, aspect ratio will be kept
  {Screenshot des Trainers}%% caption
  {Screenshot Trainer.}%% optional (short) caption for table of figures
  {fig:trainer}%% label
Die Schwierigkeit dieser Rechenaufgabe wird von Neuhold's System auf Basis der bisherigen
Rechenleistungen des Benutzers festgelegt und übermittelt. Es werden auf diesem Screen fünf Eingabefelder
präsentiert die durch eine jeweilige Berührung editierbar sind. Die drei großen Eingabefelder repräsentieren
die Eingabe des Ergebnisses für die jeweilige Stelle der Zehnerpotenz. Die darüberliegenden kleineren
Eingabefelder stellen die Überträge beim Rechnen von der jeweilig einen auf die andere Zehnerpotenz dar.
Die Position der Übertragsfelder kann je nach Wunsch über die erste Zahl oder unter die Zweite Zahl der übermittelten Rechnung 
gesetzt werden. Dies ist über das Einstellungsmenü zu definieren. Durch Berühren des roten \enquote{X Symbols} werden
alle bisher getätigten Eingabe gelöscht, und durch Berühren des grünen\enquote{Häkchen Symbols} wird die Eingabe 
bestätigt und an das Webservice übermittelt. Der/die BenutzerIn erhält sofort Feedback ob eine Rechenaufgabe
erfolgreich gelöst wurde. Abbildung \ref{fig:feedback} zeigt den Feedback Screen. 


\myfig{feedback}%% filename in figures folder
  {width=0.4\textwidth,height=0.4\textheight}%% maximum width/height, aspect ratio will be kept
  {Screenshot des Feedbacks}%% caption
  {Screenshot Feedback.}%% optional (short) caption for table of figures
  {fig:feedback}%% label

Ist die Übermittlung der Ergebnisse abgeschlossen wird sofort darauf 
eine weitere Rechenaufgabe präsentiert. Dies geschieht so lange bis der/die BenutzerIn das \enquote{Stop Symbol} berührt. 
Daraufhin landet der/die BenutzerIn wieder im Menü mit der Möglichkeit des Logouts bzw. eines erneuten 
Trainingsdurchganges.

Die Kommunikation mit dem Webservice im Zuge eines Rechendurchganges enthält mehrere Schritte, die im folgenden 
Abschnitt \ref{subsec:communication} erläutert werden. 

\subsection{Kommunikationsablauf mit Webservice}
\label{subsec:communication}

In diesem Abschnitt werden die einzelnen Schritte in der Kommunikation zwischen App und Webservice erläutert.
Abbildung \ref{fig:communicationchart} zeigt in einem abgewandelten Sequenzdiagramm wie eine erfolgreiche Kommunikation
mit dem Webservice abläuft. Ergänzend dazu werden in den folgenden Abschnitten Quelltextschnipsel
zur Erklärung des Kommunikationsablaufs verwendet.
\myfig{communicationchart}%% filename in figures folder
  {width=1\textwidth,height=1\textheight,trim=0 220 0 0}%% maximum width/height, aspect ratio will be kept
  {Ablauf der Kommunikation mit dem Webservice}%% caption
  {Kommunikation Webservice}%% optional (short) caption for table of figures
  {fig:communicationchart}%% label

\subsubsection{Anfordern einer Rechenaufgabe}
Um für einen bestimmten Benutzer eine Rechenaufgabe zu erhalten muss das Webservice am Endpoint \enquote{getProblem}
durch Übergabe der Benutzer-ID angesprochen werden. Codeausschnitt \ref{listing:getProblem} zeigt wie 
ein solcher Request aussieht. Das Feld \enquote{userID} repräsentiert die Benutzer-ID.

\begin{lstlisting}[caption=Anfordern einer Rechenaufgabe, label=listing:getProblem]
<soapenv:Envelope xmlns:xsi="http://www.w3.org/2001/XMLSchema-instance" xmlns:xsd="http://www.w3.org/2001/XMLSchema" xmlns:soapenv="http://schemas.xmlsoap.org/soap/envelope/" xmlns:web="http://plusminus.tugraz.at/webservice">
   <soapenv:Header/>
   <soapenv:Body>
      <web:getProblem soapenv:encodingStyle="http://schemas.xmlsoap.org/soap/encoding/">
         <userId xsi:type="xsd:int">1</userId>
      </web:getProblem>
   </soapenv:Body>
</soapenv:Envelope>
\end{lstlisting}

\subsubsection{Erhalten einer Rechenaufgabe}
Wurde im vorherigen Request eine gültige Benutzer-ID übergeben, bekommt man vom Webservice eine 
Antwort mit der jeweiligen Rechenaufgabe. In welcher Form eine solche Rechenaufgabe vom Webservice übermittelt
wird zeigt Codeausschnitt \ref{listing:getProblemAnswer}.

\begin{lstlisting}[caption=Erhalten einer Rechenaufgabe, label=listing:getProblemAnswer]
<SOAP-ENV:Envelope SOAP-ENV:encodingStyle="http://schemas.xmlsoap.org/soap/encoding/" xmlns:SOAP-ENV="http://schemas.xmlsoap.org/soap/envelope/" xmlns:ns1="http://plusminus.tugraz.at/webservice" xmlns:xsd="http://www.w3.org/2001/XMLSchema" xmlns:xsi="http://www.w3.org/2001/XMLSchema-instance" xmlns:SOAP-ENC="http://schemas.xmlsoap.org/soap/encoding/">
   <SOAP-ENV:Body>
      <ns1:getProblemResponse>
         <return xsi:type="ns1:WebserviceProblem">
            <id xsi:type="xsd:int">681116</id>
            <first xsi:type="xsd:int">6</first>
            <second xsi:type="xsd:int">4</second>
            <operator xsi:type="xsd:string">-</operator>
         </return>
      </ns1:getProblemResponse>
   </SOAP-ENV:Body>
</SOAP-ENV:Envelope>
\end{lstlisting}

Das Feld \enquote{id} enthält die eindeutige Identifikationsnummer der jeweiligen Rechenaufgabe die 
in weiterer Folge zur Übermittlung des Ergebnisses wieder benötigt wird. In den 
Feldern \enquote{first} bzw. \enquote{second} werden die für eine Addition bzw. Subtraktion benötigten 
Operanden übertragen und im Feld \enquote{operator} wird der Operator übermittelt, in diesem Falle entweder
\enquote{+} oder \enquote{-}. Diese Daten werden nach Erhalt geparsed und wie in Abbildung \ref{fig:trainer} in der App dargestellt.

\subsubsection{Übermitteln des Ergebnisses}
Sobald die Rechnung gelöst, und der Bestätigungs-Button berührt wurde, wird nun das Ergebnis wieder an das 
Webservice übermittelt. Codeausschnitt \ref{listing:receiveResult} zeigt die zugehörige SOAP Nachricht.
\begin{lstlisting}[caption=Übermitteln des Resultats, label=listing:receiveResult]
<soapenv:Envelope xmlns:xsi="http://www.w3.org/2001/XMLSchema-instance" xmlns:xsd="http://www.w3.org/2001/XMLSchema" xmlns:soapenv="http://schemas.xmlsoap.org/soap/envelope/" xmlns:web="http://plusminus.tugraz.at/webservice">
   <soapenv:Header/>
   <soapenv:Body>
      <web:receiveResult soapenv:encodingStyle="http://schemas.xmlsoap.org/soap/encoding/">
         <userId xsi:type="xsd:int">1</userId>
         <problemId xsi:type="xsd:int">681116</problemId>
         <result xsi:type="xsd:int">2</result>
         <carry xsi:type="xsd:int">0</carry>
         <result_pattern xsi:type="xsd:string">een</result_pattern>
         <carry_pattern xsi:type="xsd:string">ee</carry_pattern>
         <duration xsi:type="xsd:float">5</duration>
      </web:receiveResult>
   </soapenv:Body>
</soapenv:Envelope>
\end{lstlisting}

Wie bereits erwähnt enthält das Feld \enquote{userID} die Benutzer Identifikationsnummer und das 
Feld \enquote{problemID} die Rechnungs Identifikationsnummer. Im Feld \enquote{result} wird das
Ergebnis des Benutzers übermittelt. Das Feld \enquote{carry} kann maximal zwei Ziffern lang sein 
und enthält die vom Benutzer eingetragenen Zahlen in den jeweiligen Übertragsfeldern in der App.
Die Einerstelle des \enquote{carry} Feldes in der Soapnachricht enthält den Übertrag der beim Rechnen 
an der Einerstelle in der App entstanden ist und sinngemäß gilt das für die Zehnerstelle des \enquote{carry}
Feldes. Wird kein Übertrag in der App eingetragen wird automatisch die Zahl null übertragen.

Die Felder \enquote{result\_pattern} und \enquote{carry\_pattern} geben an in welches der Eingabefelder
in der App der Benutzer eine Zahl eingetragen hat, bzw ob er das Eingabefeld leer stehen gelassen hat.
Dargestellt wird das \enquote{result\_pattern} und das \enquote{carry\_pattern} als Folge der Buchstaben
\enquote{e} bzw. \enquote{n}, wobei \enquote{e} für empty und \enquote{n} für number steht. Der Buchstabe
im jeweiligen Pattern an der ganz rechten Stelle repräsentiert die Einerstelle, und der Buchstabe ganz
links repräsentiert die Hunderterstelle in der App. Als letzts Feld wird \enquote{duration} übertragen, welches
die benötigte Zeit zur Durchführung der jeweiligen Rechenaufgabe erfasst.

\subsubsection{Erhalt der Lösung}
Als Antwort auf die Übermittlung des Ergebnisses bekommt man vom Webservice die Nachricht in Codeausschnitt
\ref{listing:receiveResultResponse}. 
\begin{lstlisting}[caption=Antwort auf Übermittlung des Ergebnisses, label=listing:receiveResultResponse]
<SOAP-ENV:Envelope SOAP-ENV:encodingStyle="http://schemas.xmlsoap.org/soap/encoding/" xmlns:SOAP-ENV="http://schemas.xmlsoap.org/soap/envelope/" xmlns:ns1="http://plusminus.tugraz.at/webservice" xmlns:xsd="http://www.w3.org/2001/XMLSchema" xmlns:xsi="http://www.w3.org/2001/XMLSchema-instance" xmlns:SOAP-ENC="http://schemas.xmlsoap.org/soap/encoding/">
   <SOAP-ENV:Body>
      <ns1:receiveResultResponse>
         <return xsi:type="xsd:boolean">true</return>
      </ns1:receiveResultResponse>
   </SOAP-ENV:Body>
</SOAP-ENV:Envelope>
\end{lstlisting}

Das Feld \enquote{return} enthält entweder true oder false, je nach Erfolg.


\section{Üben}
\label{sec:excercise}

Den zweiten großen Teil der App stellt der Übungsmodus dar. Über diesen kann der/die BenutzerIn 
anonym an seinen eigenen Rechenfähigkeit arbeiten. Der Übungsmodus wird über das Hauptmenü aufgerufen 
und bietet drei Schwierigkeitsgrade zur Übung von Additionen und Subtraktionen an. 
Abbildung \ref{fig:difficulty} zeigt den Screen zur Auswahl des Schwierigkeitsgrades. Sobald eine 
der drei Schwierigkeitsgrade ausgewählt wurde, startet der Übungsmodus auch sofort.
\myfig{difficulty}%% filename in figures folder
  {width=0.4\textwidth,height=0.4\textheight}%% maximum width/height, aspect ratio will be kept
  {Screenshot des Schwierigkeitsgradauswahl im Übungsmodus}%% caption
  {Screenshot Schwierigkeitsgrad.}%% optional (short) caption for table of figures
  {fig:difficulty}%% label

Die Rechenaufgaben im Übungsmodus unterliegen einigen Einschränkungen. Diese Einschränkungen wären:
\begin{itemize}
\item Das Ergebnis der Rechenaufgabe darf nicht negativ sein
\item Das Ergebnis der Rechenaufgabe darf nicht größer als 1000 sein
\item Rechenaufgaben in der Schwierigkeitsstufe \enquote{leicht} dürfen nur maximal einstellige Operanden enthalten
\item Rechenaufgaben in der Schwierigkeitsstufe \enquote{mittel} dürfen nur maximal zweistellige Operanden enthalten
\item Rechenaufgaben in der Schwierigkeitsstufe \enquote{schwer} dürfen nur maximal dreistellige Operanden enthalten
\end{itemize}
Aufgrund dieser Einschränkungen war es notwendig bei der Erstellung der Rechenaufgaben viele 
Fälle zu untscheiden. Codeausschnit \ref{listing:problemGeneration} zeigt wie die Rechenaufgaben
generiert werden.
Zuerst wird zufällig ein erster Operand und zweiter Operand zwischen 0 und 999, und ein Operator 
generiert. Aufgrund des Schwierigkeitsgrades wird nun entschieden, ob die jeweiligen Operanden durch
100 für leichte Rechenaufgaben, durch 10 für mittelschwere Rechenaufgaben oder gar nicht für schwere
Rechenaufgaben dividiert werden. Dadurch, dass mit Integerzahlen gerechnet wird, gibt es nur ganzzahlige
Ergebnisse bei der Division und wir können das Ergebnis der Division sofort verwenden.
Weiters wird nun entschieden, dass wenn der erste generierte Operand kleiner als der zweite generierte Operand 
ist, es aufgrund unserer Einschränkungen nur eine Addition werden darf. Wenn wir nun mit einer 
Addition weiterarbeiten und das Ergebnis der Addition aber größer als 1000 wäre müssen wir eine Kombination 
zweier Operanden finden bei der die Summe der Operanden kleiner als 1000 ist. Dies geschieht in Zeile 22.
Im weiteren Verlauf in Codeausschnitt \ref{listing:problemGeneration} geht es noch darum, die generierten
Operanden in Strings umzuwandeln und aufzubereiten um sie in der App darstellen zu können.


\begin{lstlisting}[caption=Generierung von Rechenaufgaben im Übungsmodus, label=listing:problemGeneration]
    int plusMinus = (arc4random() % 2); // 0 or 1 for + or -
    int firstNum = (arc4random() % 1000); // number between 0 and 999
    int secondNum = (arc4random() % 1000); // number between 0 and 999
    
    NSUserDefaults *defaults = [NSUserDefaults standardUserDefaults];
    int difficultyLevel = [defaults integerForKey:@"excerciseDifficulty"]; //fetch difficultyLevel
    
    if(difficultyLevel == 1) { //easy
        firstNum = firstNum / 100;  
        secondNum = secondNum / 100;
    } else if (difficultyLevel == 2) {//medium
        firstNum = firstNum / 10;
        secondNum = secondNum / 10;
    }
    
    //if difficultylevel = 3, hard, nothing has to be divided
    
    if (firstNum < secondNum) //if first num is smaller than second it can only be a addition
        plusMinus = 0;         //no negative results allowed
    
    if (plusMinus == 0 && (firstNum + secondNum) >= 1000) { // if it's an addition and result > 1000
        while((firstNum + secondNum) > 1000) // find combinations of the two operands which are summed
        {                                    //smaller than 1000
            firstNum = (arc4random() % 1000);
            secondNum = (arc4random() % 1000);
        }
        
    }
    NSString *firstNumString = [NSString stringWithFormat:@"%d", firstNum]; //convert number to strings
    NSString *secondNumString = [NSString stringWithFormat:@"%d", secondNum]; //for presentation purposes
    NSString *firstOnes = 0; //initialize with 0
    NSString *firstTens = 0;
    NSString *firstHundreds = 0;
    NSString *secondOnes = 0;
    NSString *secondTens = 0;
    NSString *secondHundreds = 0;
    if ([firstNumString length] == 3) { //if first number has 3 digits
        firstOnes = [firstNumString substringWithRange:NSMakeRange(2, 1)]; // separate them correctly
        firstTens = [firstNumString substringWithRange:NSMakeRange(1, 1)];
        firstHundreds = [firstNumString substringWithRange:NSMakeRange(0, 1)];
        
    } else if ([firstNumString length] == 2) { // if first number has 2 digits
        firstOnes = [firstNumString substringWithRange:NSMakeRange(1, 1)]; // separate them correctly
        firstTens = [firstNumString substringWithRange:NSMakeRange(0, 1)];
        firstHundreds = @""; //and empty hundreds field
    } else {// if first number digits
        
        firstOnes = [firstNumString substringWithRange:NSMakeRange(0, 1)]; // separate them correctly
        firstTens = @"";//and tens hundreds field
        firstHundreds = @"";//and empty hundreds field
    }
    
    //same stuff for the second number
    
    if ([secondNumString length] == 3) {
        secondOnes = [secondNumString substringWithRange:NSMakeRange(2, 1)];
        secondTens = [secondNumString substringWithRange:NSMakeRange(1, 1)];
        secondHundreds = [secondNumString substringWithRange:NSMakeRange(0, 1)];
        
    } else if ([secondNumString length] == 2) {
        secondOnes = [secondNumString substringWithRange:NSMakeRange(1, 1)];
        secondTens = [secondNumString substringWithRange:NSMakeRange(0, 1)];
        secondHundreds = @"";
    } else {
        
        secondOnes = [secondNumString substringWithRange:NSMakeRange(0, 1)];
        secondTens = @"";
        secondHundreds = @"";
    }

    
    //fill screen in app
    self.firstOnes.text = firstOnes;
    self.firstTens.text = firstTens;
    self.firstHundreds.text = firstHundreds;
    
    self.secondOnes.text = secondOnes;
    self.secondTens.text = secondTens;
    self.secondHundreds.text = secondHundreds;
    
    //insert + or -
    self.plusMinus.text = (plusMinus == 0) ? @"+" : @"-";
    //calculate desired result for evaluation purposes
    self.desiredResult = (plusMinus == 0) ? (firstNum + secondNum) : (firstNum - secondNum);
\end{lstlisting}

Die Darstellung der Rechenaufgaben im Übungsmodus sieht grundsätzlich gleich aus wie die Darstellung 
im Trainingsmodus in Abbildung \ref{fig:trainer}. Im Unterschied zum Trainingsmodus bekommt der/die
BenutzerIn nach Beendigung des Übungsmodus durch Berühren des Stop-Buttons eine Auswertung seiner 
Übungsleistung dargestellt. Dazu wird die Anzahl der korrekt gelösten Übungsrechenaufgaben mitgeloggt.
Abbildung \ref{fig:evaluation} zeigt eine beispielhafte Auswertung eines Übungsdurchlaufs.

\myfig{evaluation}%% filename in figures folder
  {width=0.4\textwidth,height=0.4\textheight}%% maximum width/height, aspect ratio will be kept
  {Screenshot des Auswertungsscreens im Übungsmodus}%% caption
  {Screenshot Auswertung.}%% optional (short) caption for table of figures
  {fig:evaluation}%% label

\section{Einstellungen}
\label{sec:settings}
Unter Einstellungen im Hauptmenü hat der/die BenutzerIn die Möglichkeit die Position des Übertragfeldes
im Trainings -und Übungsmodus zu verändern. Im englischsprachigen Raum ist es üblich das Übertragfeld
ober dem ersten Operanden zu positionieren. Im deutschsprachigen Raum ist das Übertragfeld gewöhnlich
unter dem zweiten Operanden zu finden. Abbildung \ref{fig:settings} zeigt den Einstellungsscreen.
\myfig{settings}%% filename in figures folder
  {width=0.4\textwidth,height=0.4\textheight}%% maximum width/height, aspect ratio will be kept
  {Screenshot des Einstellungsscreens}%% caption
  {Screenshot Einstellungen.}%% optional (short) caption for table of figures
  {fig:settings}%% label



\section{Hilfe}
\label{sec:help}

Im Menüpunkt Hilfe bekommt der/die BenutzerIn eine kurze Erklärung der Funktionalität der App.
Abbildung \ref{fig:help} zeigt den Hilfescreen.
\myfig{help}%% filename in figures folder
  {width=0.4\textwidth,height=0.4\textheight}%% maximum width/height, aspect ratio will be kept
  {Screenshot des Hilfescreens}%% caption
  {Screenshot Hilfe.}%% optional (short) caption for table of figures
  {fig:help}%% label
  

  





%% vim:foldmethod=expr
%% vim:fde=getline(v\:lnum)=~'^%%%%\ .\\+'?'>1'\:'='
%%% Local Variables: 
%%% mode: latex
%%% mode: auto-fill
%%% mode: flyspell
%%% eval: (ispell-change-dictionary "en_US")
%%% TeX-master: "main"
%%% End: 
