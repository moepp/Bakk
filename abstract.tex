%%%% Time-stamp: <2013-02-25 10:31:01 vk>


\chapter*{Kurzfassung}
\label{cha:abstract}



Im Zuge meiner Bachelorarbeit, wurde eine native iPhone App als Ergänzung
zu Benedikt Neuholds Additions- und Subtraktionstrainer entwickelt. 
Der Funktionsumfang besteht grundsätzlich aus 2 Teilen: 

Als Primärfunktion wurde ein Online-Trainer entwickelt, der per Webservice abfrägt 
ob ein User Zugriff auf das System hat oder nicht, und der bei erfolgreicher Anmeldung beim 
Webservice, die für den User bestimmten Rechenaufgaben übermittelt bekommt.
Diese Rechenaufgaben werden durch die App in grafisch ansprechender Weise präsentiert,
und der/die BenutzerIn hat die Möglichkeit das Ergebnis einzugeben.
Für die Auswertung der Rechenaufgaben werden die Ergebnisse, und auch alle Zwischenergebnisse 
in Form von Überträgen, mitgeloggt und nach Abschluss des Rechendurchlaufes wieder 
an das Webservice übermittelt wo das Ergebnis und der Lernfortschritt gespeichert wird.

Die Sekundärfunktion der App ist eine Offline-Übungsmöglichkeit, die in unterschiedlichen 
Schwierigkeitsstufen unauthorisiert/anonym durchführbar ist, und dem Zwecke der Verbesserung
der Rechenfähigkeiten des Users dient. 



%\glsresetall %% all glossary entries should be used in long form (again)
%% vim:foldmethod=expr
%% vim:fde=getline(v\:lnum)=~'^%%%%\ .\\+'?'>1'\:'='
%%% Local Variables:
%%% mode: latex
%%% mode: auto-fill
%%% mode: flyspell
%%% eval: (ispell-change-dictionary "en_US")
%%% TeX-master: "main"
%%% End:
