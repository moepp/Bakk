%%%% Time-stamp: <2013-09-24 14:39:01 vk>
%% ========================================================================
%%%% Disclaimer
%% ========================================================================
%%
%% created by
%%
%%      Stephan Moser
%%

%% ========================================================================
%%%% State of the art
%% ========================================================================

%% example text content
%% scrartcl and scrreprt starts with section, subsection, subsubsection, ...
%% scrbook starts with part (optional), chapter, section, ...\chapter{State of the Art}
\chapter{Stand der Technik}
\label{chap:sota}

In diesem Kapitel werden Arbeiten zum Thema 
\enquote{Addition und Subtraktion mit mobilen Geräten} vorgestellt.
Dabei handelt es sich vorwiegend um aktuelle iPhone Apps aus 
Apples' iTunes Store. \footcite{https://itunes.apple.com/de/genre/ios/id36?mt=8} Diese Apps sind 
gewöhnlich für Kinder im Pflichtschulalter gedacht und dadurch auch meist grafisch ansprechend 
designt.

In den folgenden Abschnitten werden ein paar ausgewählte Apps vorgestellt.

\section{MathBoard \footcite{https://itunes.apple.com/de/app/mathboard/id373909837?mt=8}}
Diese App dient als Best Practice App im Bereich Mathematik. Aufgrunddessen wird sie 
auch von Apple selbst bei diversen Veranstaltungen präsentiert. In Abbildung \ref{fig:mathboard} 
wird ein Screenshot dieser App gezeigt, auf dem sich aber erkennen lässt, dass der Funktionsumfang dieser App 
nicht wirklich mit der App die in dieser Arbeit präsentiert wird korreliert, und deswegen hier nur
als \enquote{Best Practice} Beispiel angeführt wird.
\myfig{mathboard}%% filename in figures folder
  {width=0.4\textwidth,height=0.4\textheight}%% maximum width/height, aspect ratio will be kept
  {Screenshot von Mathboard.}%% caption
  {Screenshot Mathboard}%% optional (short) caption for table of figures
  {fig:mathboard}%% label

\section{Addition Master: Mathematik Spiel \footcite{https://itunes.apple.com/de/app/addition-master-mathematik/id672669932?mt=8}}
In Abbildung \ref{fig:addmaster} ist ersichtlich, dass die Benutzeroberfläche dieser App, 
und dabei vor allem die Präsentation der Zahleneingabemöglichkeit sehr ähnlich der in dieser Arbeit 
vorgestellten App gestaltet wurde. Zum Funktionsumfang gehören hier: 
\begin{itemize}
	\item Trainingsmodus
	\item Statistik
	\item Übungsmodus
\end{itemize}

\myfig{addmaster}%% filename in figures folder
  {width=0.4\textwidth,height=0.4\textheight}%% maximum width/height, aspect ratio will be kept
  {Screenshot von Addition Master: Mathematik Spiel.}%% caption
  {Screenshot Addition Master}%% optional (short) caption for table of figures
  {fig:addmaster}%% label



\section{Addition ! \footcite{https://itunes.apple.com/de/app/addition-!/id447548669?mt=8} and Subtraction !\footcite{https://itunes.apple.com/de/app/subtraction-!/id447548515?mt=8}}
Hierbei handelt es sich um zwei separat existierende Apps vom selben Entwickler zum Thema
Addition und Subtraktion. In der Recherche waren diese zwei Apps auch die einzigen, bei denen
der/die SchülerIn Überträge zur Rechenerleichterung notieren konnte.
Funktionalität:
\begin{itemize}
	\item 2 oder 3 Summanden
	\item bis 6 Ziffern pro Summand
	\item Hilfe zur Problemstellung
	\item Tipp zur Problemstellung
	\item Tutorial in dem die App erklärt wird
	\item Editor für eigene Problemstellungen
\end{itemize}
In Abbildung \ref{fig:addition!} ist ein Screenshot der App \enquote{Addition !} zu sehen. Darauf ist ersichtlich, dass
die Überträge über dem ersten Summanden einzutragen sind. Überträge über dem ersten Summanden
zu notieren werden ist jedoch nur im englischsprachigen Raum üblich, im deutschsprachigen Raum werden 
die Überträge üblicherweise unter dem letzten Summanden notiert. In der in dieser Arbeit vorgestellten
App ist es möglich die Felder für die Überträge entweder oben oder unten anzeigen zu lassen.
\myfig{addition!}%% filename in figures folder
  {width=0.4\textwidth,height=0.4\textheight}%% maximum width/height, aspect ratio will be kept
  {Screenshot von Addition !.}%% caption
  {Screenshot Addition !}%% optional (short) caption for table of figures
  {fig:addition!}%% label

Abbildung \ref{fig:subtraction!} zeigt einen Screenshot der App \enquote{Subtraction !}. Dabei ist 
eine ausgeklügelte Methode zur Notierung der Überträge bei Subtraktionen ersichtlich.
\myfig{subtraction!}%% filename in figures folder
  {width=0.4\textwidth,height=0.4\textheight}%% maximum width/height, aspect ratio will be kept
  {Screenshot von Subtraction !.}%% caption
  {Screenshot Subtraction !}%% optional (short) caption for table of figures
  {fig:subtraction!}%% label


\section{Weitere Apps}

In diesem Abschnitt werden kurz weitere ausgewählte Apps im Bereich des mobilen Lernens vorgestellt.

\myfig{add&sub}%% filename in figures folder
  {width=0.4\textwidth,height=0.4\textheight}%% maximum width/height, aspect ratio will be kept
  {Screenshot von Add \& Sub.}%% caption
  {Screenshot Add \& Sub}%% optional (short) caption for table of figures
  {fig:addandsub}%% label
Abbildung \ref{fig:addandsub} zeigt die App \enquote{Add \& Sub \footcite{https://itunes.apple.com/de/app/add-sub/id693077439?mt=8}}. Sie ist sehr einfach gehalten und auch 
in ihrem Funktionsumfang eingeschränkt.

\myfig{addsubspringbird}%% filename in figures folder
  {width=0.4\textwidth,height=0.4\textheight}%% maximum width/height, aspect ratio will be kept
  {Screenshot von Add \& Sub with Springbird.}%% caption
  {Screenshot Add \& Sub with Springbird}%% optional (short) caption for table of figures
  {fig:addandsubspringbird}%% label

\myfig{addsubforkids}%% filename in figures folder
  {width=0.4\textwidth,height=0.4\textheight}%% maximum width/height, aspect ratio will be kept
  {Screenshot von Add \& Sub For Kids.}%% caption
  {Screenshot Add \& Sub For Kids}%% optional (short) caption for table of figures
  {fig:addsubforkids}%% label

Eine weitere Möglichkeit Mathematik Apps für Kinder attraktiv zu gestalten ist, die Apps als Spiele
aufzubauen. Die Abbildungen \ref{fig:addandsubspringbird} und \ref{fig:addsubforkids} zeigen die Apps
\enquote{Add \& Sub with Springbird \footcite{https://itunes.apple.com/de/app/add-subtract-springbird-mathe/id601505771?mt=8}}
und
\enquote{Addition \& Subtraction For Kids \footcite{https://itunes.apple.com/de/app/addition-subtraction-for-kids/id426907035?mt=8}}
die vor allem für SchülerInnen bis 10 Jahren auf dieses Prinzip setzt.


Weiters zu erwähnen sind die Apps:
\begin{itemize}
\item \enquote{Add Sub K-1 \footcite{https://itunes.apple.com/de/app/add-sub-k-1/id486199509?mt=8}}
in Abbildung \ref{fig:addsubk1}
\item \enquote{Addition - Subtraction\footcite{https://itunes.apple.com/de/app/addition-subtraction/id542109601?mt=8}}
in Abbildung \ref{fig:add-sub}
\item \enquote{Subtract with Fun \footcite{https://itunes.apple.com/de/app/subtract-with-fun/id699563137?mt=8}}
in Abbildung \ref{fig:subwithfun}

\end{itemize}

\myfig{addsubk1}%% filename in figures folder
  {width=0.4\textwidth,height=0.4\textheight}%% maximum width/height, aspect ratio will be kept
  {Screenshot von Add Sub K-1.}%% caption
  {Screenshot Add Sub K-1}%% optional (short) caption for table of figures
  {fig:addsubk1}%% label

\myfig{add-sub}%% filename in figures folder
  {width=0.4\textwidth,height=0.4\textheight}%% maximum width/height, aspect ratio will be kept
  {Screenshot von Addition - Subtraction.}%% caption
  {Screenshot Addition - Subtraction}%% optional (short) caption for table of figures
  {fig:add-sub}%% label
  
\myfig{subwithfun}%% filename in figures folder
  {width=0.4\textwidth,height=0.4\textheight}%% maximum width/height, aspect ratio will be kept
  {Screenshot von Subtraction ith Fun.}%% caption
  {Screenshot Subtraction with Fun}%% optional (short) caption for table of figures
  {fig:subwithfun}%% label



%% vim:foldmethod=expr
%% vim:fde=getline(v\:lnum)=~'^%%%%\ .\\+'?'>1'\:'='
%%% Local Variables: 
%%% mode: latex
%%% mode: auto-fill
%%% mode: flyspell
%%% eval: (ispell-change-dictionary "en_US")
%%% TeX-master: "main"
%%% End: 
