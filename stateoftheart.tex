%%%% Time-stamp: <2013-09-24 14:39:01 vk>
%% ========================================================================
%%%% Disclaimer
%% ========================================================================
%%
%% created by
%%
%%      Stephan Moser
%%

%% ========================================================================
%%%% State of the art
%% ========================================================================

%% example text content
%% scrartcl and scrreprt starts with section, subsection, subsubsection, ...
%% scrbook starts with part (optional), chapter, section, ...\chapter{State of the Art}
\chapter{Stand der Technik}
\label{chap:sota}

In diesem Kapitel werden Arbeiten zum Thema 
\enquote{Addition und Subtraktion mit mobilen Geräten} vorgestellt.
Dabei handelt es sich vorwiegend um aktuelle iPhone Apps aus 
Apple's iTunes Store. \footnote{\url{https://itunes.apple.com/de/genre/ios/id36?mt=8} (letzter Zugriff 30. 10. 2013)} Diese Apps sind 
gewöhnlich für Kinder im Pflichtschulalter gedacht und dadurch auch meist grafisch ansprechend 
gestaltet.

In den folgenden Abschnitten werden ein paar ausgewählte Apps vorgestellt.

\section[MathBoard]{MathBoard\footnote{\url{https://itunes.apple.com/de/app/mathboard/id373909837?mt=8} (letzter Zugriff 30. 10. 2013)}}
Diese App dient als \enquote{Best Practice} App im Bereich Mathematik. Aufgrund dessen wird sie 
auch von Apple selbst bei diversen Veranstaltungen präsentiert. In Abbildung \ref{fig:mathboard} 
wird ein Screenshot dieser App gezeigt, auf dem sich aber bereits erkennen lässt, dass das Ziel dieser App 
nicht wirklich mit dem Ziel der in dieser Arbeit präsentierten App korreliert, und nur deswegen hier als 
Beispiel angeführt wird, weil Apple diese App forciert.

MathBoard ist eine konfigurierbare Mathe App,geeignet für alle Altersgruppen vom Kindergarten (mit leichten Additions- und Subtraktionsproblemen) 
bis zur Grundschule, wo Multiplikation und Dividieren schwieriger werden. MathBoard ermöglicht 
den Fähigkeiten des Kindes angepasste Rechenaufgaben zu stellen.

Folgende Features sind in MathBoard verfügbar:

\begin{itemize}
	\item Mathematik Probleme nach dem Zufallsprinzip für Addition, Subtraktion, Multiplikation, Dividieren, Quadrate, Kubik und Quadratwurzeln.
	\item Zahlenbereiche sind konfigurierbar, einschliesslich der Möglichkeit bestimmte Zahlen in jedem Problem zu erfragen, wie auch negative Antworten wegzulassen.
	\item Anzahl und Reihenfolge von dargestellten Ziffern können auf bestimmte Lernstufen begrenzt werden, um die Probleme den Lernlevel anzupassen (z.B. 2-stellige Nummern über 1-stellige Nummern).
	\item Erstellt einfache Probleme, wie auch Einzel Schritt algebraische Gleichungen.
	\item Umfasst sowohl Multiple-Choice, als auch das Ausfüllen der Lücken.
	\item Für die Lösung der Aktivitäten und der Quizes kann ein bestimmter Zeitumfang festgelegt werden.
	\item Konfigurationseinstellungen der Probleme können für spätere Verwendung gespeichert, sowie anderen mitgeteilt werden.
	\item Unterstützung für mehrere Studenten Profile ist möglich. 
	\item Der eingeschlossene Problemlöser beschreibt die notwendigen Schritte um Additions-, Subtraktions-, Multiplikations- und Divisionsprobleme zu lösen.
	\item Folgende zusätzliche Aktivitäten sind eingeschlossen (Finde das Zeichen, Gleichung/Ungleichung und Mathe Match).
	\item Nützliche Referenz Mathematik Tabellen zum Zählen, für Addition, Subtraktion und Multiplikation.
\end{itemize}

\myfig{mathboard}%% filename in figures folder
  {width=0.4\textwidth,height=0.4\textheight}%% maximum width/height, aspect ratio will be kept
  {Screenshot von Mathboard.}%% caption
  {Screenshot Mathboard}%% optional (short) caption for table of figures
  {fig:mathboard}%% label

\section[Addition Master: Mathematik Spiel]{Addition Master: Mathematik Spiel\footnote{\url{https://itunes.apple.com/de/app/addition-master-mathematik/id672669932?mt=8 } (letzter Zugriff 30. 10. 2013)}}
In Abbildung \ref{fig:addmaster} ist ersichtlich, dass die Benutzeroberfläche dieser App, 
und dabei vor allem die Präsentation der Zahleneingabemöglichkeit sehr ähnlich der in dieser Arbeit 
vorgestellten App gestaltet wurde. Zum Funktionsumfang gehören hier: 
\begin{itemize}
	\item Trainingsmodus
	\item Statistik
	\item Übungsmodus
\end{itemize}

Der Name dieser App ist hier etwas irreführend, da sie weniger ein Spiel als mehr ein Trainingsprogramm
darstellt.

\myfig{addmaster}%% filename in figures folder
  {width=0.4\textwidth,height=0.4\textheight}%% maximum width/height, aspect ratio will be kept
  {Screenshot von Addition Master: Mathematik Spiel.}%% caption
  {Screenshot Addition Master}%% optional (short) caption for table of figures
  {fig:addmaster}%% label



\section[Addition! and Subtraction!]{Addition!\footnote{\url{https://itunes.apple.com/de/app/addition-!/id447548669?mt=8} (letzter Zugriff 30. 10. 2013)} and Subtraction!\footnote{\url{https://itunes.apple.com/de/app/subtraction-!/id447548515?mt=8} (letzter Zugriff 30. 10. 2013)}}
Hierbei handelt es sich um zwei separat existierende Apps vom selben Entwickler zum Thema
Addition und Subtraktion. In der Recherche waren diese zwei Apps auch die einzigen, bei denen
der/die SchülerIn Überträge zur Rechenerleichterung notieren konnte.
Funktionalität:
\begin{itemize}
	\item 2 oder 3 Summanden
	\item bis 6 Ziffern pro Summand
	\item Hilfe zur Problemstellung
	\item Tipp zur Problemstellung
	\item Tutorial in dem die App erklärt wird
	\item Editor für eigene Problemstellungen
\end{itemize}
In Abbildung \ref{fig:addition!} ist ein Screenshot der App \enquote{Addition !} zu sehen. Darauf ist ersichtlich, dass
die Überträge über dem ersten Summanden einzutragen sind. Überträge über dem ersten Summanden
zu notieren ist jedoch nur im englischsprachigen Raum üblich, im deutschsprachigen Raum werden 
die Überträge üblicherweise unter dem letzten Summanden notiert. In der in dieser Arbeit vorgestellten
App ist es möglich die Felder für die Überträge entweder oben oder unten anzeigen zu lassen.
\myfig{addition!}%% filename in figures folder
  {width=0.4\textwidth,height=0.4\textheight}%% maximum width/height, aspect ratio will be kept
  {Screenshot von Addition !.}%% caption
  {Screenshot Addition !}%% optional (short) caption for table of figures
  {fig:addition!}%% label

Abbildung \ref{fig:subtraction!} zeigt einen Screenshot der App \enquote{Subtraction !}. Dabei ist 
eine ausgeklügelte Methode zur Notierung der Überträge bei Subtraktionen ersichtlich.
\myfig{subtraction!}%% filename in figures folder
  {width=0.4\textwidth,height=0.4\textheight}%% maximum width/height, aspect ratio will be kept
  {Screenshot von Subtraction !.}%% caption
  {Screenshot Subtraction !}%% optional (short) caption for table of figures
  {fig:subtraction!}%% label


\section{Weitere Apps}

In diesem Abschnitt werden kurz weitere ausgewählte Apps im Bereich des mobilen Lernens vorgestellt.

\myfig{add&sub}%% filename in figures folder
  {width=0.4\textwidth,height=0.4\textheight}%% maximum width/height, aspect ratio will be kept
  {Screenshot von Add \& Sub.}%% caption
  {Screenshot Add \& Sub}%% optional (short) caption for table of figures
  {fig:addandsub}%% label
Abbildung \ref{fig:addandsub} zeigt die App \enquote{Add \& Sub \footnote{\url{https://itunes.apple.com/de/app/add-sub/id693077439?mt=8} (letzter Zugriff 30. 10. 2013)}}. Sie ist sehr einfach gehalten und auch 
in ihrem Funktionsumfang eingeschränkt.

\myfig{addsubspringbird}%% filename in figures folder
  {width=0.4\textwidth,height=0.4\textheight}%% maximum width/height, aspect ratio will be kept
  {Screenshot von Add \& Sub with Springbird.}%% caption
  {Screenshot Add \& Sub with Springbird}%% optional (short) caption for table of figures
  {fig:addandsubspringbird}%% label

\myfig{addsubforkids}%% filename in figures folder
  {width=0.4\textwidth,height=0.4\textheight}%% maximum width/height, aspect ratio will be kept
  {Screenshot von Add \& Sub For Kids.}%% caption
  {Screenshot Add \& Sub For Kids}%% optional (short) caption for table of figures
  {fig:addsubforkids}%% label

Eine weitere Möglichkeit Mathematik Apps für Kinder attraktiv zu gestalten ist, die Apps als Spiele
aufzubauen. Die Abbildungen \ref{fig:addandsubspringbird} und \ref{fig:addsubforkids} zeigen die Apps
\enquote{Add \& Sub with Springbird \footnote{\url{https://itunes.apple.com/de/app/add-subtract-springbird-mathe/id601505771?mt=8} (letzter Zugriff 30. 10. 2013)}}
und
\enquote{Addition \& Subtraction For Kids \footnote{\url{https://itunes.apple.com/de/app/addition-subtraction-for-kids/id426907035?mt=8} (letzter Zugriff 30. 10. 2013)}}
die vor allem für SchülerInnen bis 10 Jahren auf dieses Prinzip setzt.


Weiters zu erwähnen sind die Apps:
\begin{itemize}
\item \enquote{Add Sub K-1 \footnote{\url{https://itunes.apple.com/de/app/add-sub-k-1/id486199509?mt=8} (letzter Zugriff 30. 10. 2013)}}
in Abbildung \ref{fig:addsubk1}
\item \enquote{Addition - Subtraction\footnote{\url{https://itunes.apple.com/de/app/addition-subtraction/id542109601?mt=8} (letzter Zugriff 30. 10. 2013)}}
in Abbildung \ref{fig:add-sub}
\item \enquote{Subtract with Fun \footnote{\url{https://itunes.apple.com/de/app/subtract-with-fun/id699563137?mt=8} (letzter Zugriff 30. 10. 2013)}}
in Abbildung \ref{fig:subwithfun}

\end{itemize}

\myfig{addsubk1}%% filename in figures folder
  {width=0.4\textwidth,height=0.4\textheight}%% maximum width/height, aspect ratio will be kept
  {Screenshot von Add Sub K-1.}%% caption
  {Screenshot Add Sub K-1}%% optional (short) caption for table of figures
  {fig:addsubk1}%% label

\myfig{add-sub}%% filename in figures folder
  {width=0.4\textwidth,height=0.4\textheight}%% maximum width/height, aspect ratio will be kept
  {Screenshot von Addition - Subtraction.}%% caption
  {Screenshot Addition - Subtraction}%% optional (short) caption for table of figures
  {fig:add-sub}%% label
  
\myfig{subwithfun}%% filename in figures folder
  {width=0.4\textwidth,height=0.4\textheight}%% maximum width/height, aspect ratio will be kept
  {Screenshot von Subtraction with Fun.}%% caption
  {Screenshot Subtraction with Fun}%% optional (short) caption for table of figures
  {fig:subwithfun}%% label

\section{Zusammenfassung}

Aufgrund der steigenden Anzahl von Schülern und Jugendlichen die im Besitz eines Smartphones sind, 
wird Mobile Learning immer mehr zu einem integralen Bestandteil der Erziehung. Durch die leicht zu 
bedienenden Interfaces auf Smartphones und Tablets ist es sogar für die jüngsten Kinder sehr einfach
sofort damit zu interagieren. Mobile Geräte eröffnen Tore zum Lernen, zur Zusammenarbeit und zur 
Produktivität. Eine der am schnellsten wachsenden Facetten hierbei sind mobile Apps.

Die Recherche zu mathematischen Lernapps für Kinder hat gezeigt, dass der mit dieser App
eingeschlagene Weg, bei dem alle Ergebnisse protokolliert und analysiert werden, und auf Basis dessen
maßgeschneiderte Rechenbeispiele generiert werden, in der Praxis jedoch noch nicht sehr häufig anzufinden 
ist. Im Vergleich zur der in dieser Arbeit präsentierten App könnte man die Vorgangsweise zur 
Generierung der Rechenbeispiele bei den untersuchten Apps eher als \enquote{statisch} bezeichnen, 
denn sie verlangen meist einen Input vom User in Form eines Schwierigkeitsgrades oder ähnlichem, 
um den Bedürfnissen des Benutzers entsprechende Rechenbeispiele zu liefern. 
Entsprechend der getätigten Aussagen im Horizon Report des \citet{NMC2013} : \enquote{the demand for
personalized learning is not adequately supported by current technology or practices} und 
\enquote{The notion that onesize-fits-all teaching methods are neither effective nor acceptable for 
today’s diverse students is generally accepted among K-12 educators.} ist die mit 
dieser App getätigte Arbeit auf jeden Fall noch immer als Forschungsarbeit in diesem Bereich anzusehen.


%% vim:foldmethod=expr
%% vim:fde=getline(v\:lnum)=~'^%%%%\ .\\+'?'>1'\:'='
%%% Local Variables: 
%%% mode: latex
%%% mode: auto-fill
%%% mode: flyspell
%%% eval: (ispell-change-dictionary "en_US")
%%% TeX-master: "main"
%%% End: 
